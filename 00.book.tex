 \documentclass[pdftex,12pt, oneside]{article}

%\usepackage[paperwidth=8.5in, paperheight=13in]{geometry} % Folio
\usepackage[paperwidth=8.27in, paperheight=11.69in]{geometry} % A4

\usepackage{makeidx}         % allows index generation
\usepackage{graphicx}        % standard LaTeX graphics tool
                             % when including figure files
\usepackage[bottom]{footmisc}% places footnotes at page bottom
\usepackage[english]{babel}
\usepackage{enumerate}
\usepackage{paralist}
\usepackage{float}
\usepackage{gensymb}  
\usepackage{listings}
\usepackage{color}
\usepackage{mathtools} % atau \usepackage{amsmath}
\renewcommand{\baselinestretch}{1.5}

\newcommand{\HRule}{\rule{\linewidth}{0.5mm}}

\definecolor{codegreen}{rgb}{0,0.6,0}
\definecolor{codegray}{rgb}{0.5,0.5,0.5}
\definecolor{codepurple}{rgb}{0.58,0,0.82}
\definecolor{backcolor}{rgb}{0.95,0.95,0.92}

\lstdefinestyle{mystyle}{
  backgroundcolor=\color{backcolor},
  commentstyle=\color{codegreen},
  keywordstyle=\color{magenta},
  stringstyle=\color{codepurple},
  basicstyle=\footnotesize,
  breakatwhitespace=false,
  breaklines=true,
  captionpos=b,
  keepspaces=true,
  numbers=left,
  numbersep=5pt,
  showspaces=false,
  showstringspaces=false,
  showtabs=false,
  tabsize=2
}

\lstset{style=mystyle}


\begin{document}
\sloppy % biar section ga melebar melewati kertas

\begin{center}
{\large IMPLEMENTASI RANCANGAN SISTEM BASIS DATA - WS PBB}
\\[1cm]
XX April 2017\\
Priyanto Tamami, S.Kom.
\end{center}

%\frontmatter%%%%%%%%%%%%%%%%%%%%%%%%%%%%%%%%%%%%%%%%%%%%%%%%%%%%%%


%%%%%%%%%%%%%%%%%%%%%%%%%%%%%%%%%%%%%%%%%%%%%%%%%%%%%%%%%%%%%%%%%%%%%%

\section{TAHAPAN IMPLEMENTASI}

Tahapan implementasi dari sistem basis data untuk aplikasi \textit{Web Service} PBB adalah sebagai berikut :

\begin{enumerate}[1.]
  \item Telaah Ulang Rancangan Sistem Basis Data
  \item Penjadwalan Tugas Pengembangan Sistem Basis Data
  \item \textit{Coding} Program
  \item Pengujian Sistem Basis Data
\end{enumerate}

Tahapan secara rinci akan dijelaskan sebagai berikut.

\subsection{Telaah Ulang Rancangan}

Penelaahan ulang dilakukan melalui alur data yang terjadi pada saat aplikasi digunakan nantinya. Kebutuhan akan sistem basis data dapat ditelaah berdasarkan diagram \textit{use-case} seperti pada gambar \ref{fig:use-case} :

\begin{figure}[H]
	\centering
	\includegraphics[width=0.5\textwidth]{./resources/001-uml-use-case}
	\caption{Diagram \textit{Use-Case}}
	\label{fig:use-case}
\end{figure}

Pada diagram tersebut, aplikasi nantinya akan melayani 3 (tiga) hal, yaitu \textit{inquiry}, pencatatan pembayaran, dan \textit{reversal} pembayaran. Kebutuhan rinci untuk tiap aktivitas tersebut adalah sebagai berikut :

\begin{enumerate}[1.]
	\item \textit{Inquiry} Data
	
	Pada aktivitas \textit{inquiry} data, tabel yang dibutuhkan adalah tabel SPPT sebagai tabel utama yang memuat informasi nama wajib pajak, tahun pajak, dan besarnya PBB-P2 terhutang. Tabel lain yang diperlukan adalah tabel REF\_KECAMATAN dan REF\_KELURAHAN yang menyimpan informasi nama Kecamatan dan nama Desa/Kelurahan. 
	
	\item Pencatatan Pembayaran
	
	Pada aktivitas pencatatan pembayaran SPPT PBB-P2, tabel yang dibutuhkan atau digunakan adalah sebagai berikut :
	
	\begin{itemize}
		\item Tabel SPPT sebagai tabel yang menyimpan informasi utama dari tagihan tahun pajak tertentu, nantinya \textit{field} STATUS\_PEMBAYARAN\_SPPT pada tabel ini akan diubah menjadi 1 (satu) apabila proses pencatatan pembayaran berhasil.
		\item Tabel PEMBAYARAN\_SPPT digunakan sebagai tempat untuk mencatatkan transaksi pembayaran sesuai aplikasi SISMIOP. 
		\item Tabel DAT\_OP\_BUMI digunakan untuk mengambil informasi kode zona nilai tanah dan nilai sistem bumi (tanah) sebagai bahan penyusunan kode Nomor Transaksi Pajak Daerah (NTPD).
		\item Tabel REF\_KECAMATAN dan REF\_KELURAHAN yang digunakan untuk menampilkan informasi nama Kecamatan dan nama Desa/Kelurahan sebagai bahan informasi pencetakan Surat Setoran Pajak Daerah oleh pihak Bank / Tempat Pembayaran.
		\item Tabel LOG\_TRX\_PEMBAYARAN yang digunakan sebagai tempat penyimpanan aktivitas pencatatan transaksi yang telah terjadi.
	\end{itemize} 
	
	\item \textit{Reversal} Pembayaran
	
	Pada aktivitas \textit{reversal} pembayaran SPPT PBB-P2, dimaksudkan apabila ada kesalahan pencatatan pembayaran yang telah terjadi sebelumnya, tagihannya dapat dikembalikan ke status terhutang kembali. Adapun tabel yang digunakan untuk aktivitas ini adalah sebagai berikut :
	
	\begin{itemize}
		\item Tabel SPPT, tabel ini digunakan agar aplikasi dapat mengubah / memperbarui \textit{field} STATUS\_PEMBAYARAN\_SPPT menjadi 0 (nol) yang artinya Nomor Objek Pajak (NOP) untuk tahun pajak tercantum belum dibayarkan dan dalam status terhutang.
		\item Tabel PEMBAYARAN\_SPPT, tabel ini digunakan agar aplikasi dapat menghapus data terakhir saat terjadinya pembayaran.
		\item Tabel LOG\_TRX\_PEMBAYARAN, tabel ini digunakan oleh aplikasi untuk melakukan pencocokan data Nomor Transaksi Pajak Daerah (NTPD) yang menunjukkan bahwa transaksi yang akan dilakukan \textit{reversal} adalah transaksi yang pernah dilakukan pencatatan pembayaran sebelumnya.
		\item Tabel LOG\_REVERSAL, tabel ini digunakan oleh aplikasi untuk menyimpan catatan aktivitas \textit{reversal} yang telah terjadi.
	\end{itemize}
\end{enumerate}

\subsection{Penjadwalan tugas pengembangan}

Jadwal tugas pengembangan untuk membangun sistem basis data pada aplikasi \textit{Web Service} dilakukan oleh 1 (satu) orang fungsional Pranata Komputer dan dapat diselesaikan dalam 1 hari saja karena struktur tabel pada basis data sebagian menggunakan sistem yang sudah ada (dari SISMIOP), sehingga hanya diperlukan tabel LOG\_REVERSAL dan tabel LOG\_TRX\_PEMBAYARAN saja. Adapun \textit{store procedures} yang dibangun juga cukup sederhana.

\subsection{Coding Program (Pembuatan database, tabel, relasi tabel, indeks, dan trigger)}

Bagian \textit{coding} program akan terdiri dari beberapa hal, yaitu :

\begin{enumerate}[1.]
	\item Pembuatan Database
	
	Pembuatan basis data baru tidak diperlukan, karena sistem aplikasinya akan menggunakan basis data yang sudah ada agar dapat langsung dilihat hasilnya pada aplikasi SISMIOP.
	
	\item Pembuatan Tabel
	
	Penambahan tabel yang diperlukan untuk sistem aplikasi \textit{Web Service} PBB-P2 ini hanya 2 (dua) tabel, yaitu tabel LOG\_TRX\_PEMBAYARAN dan tabel LOG\_REVERSAL. Kode untuk membuat tabel tersebut adalah sebagai berikut :
	
	\begin{enumerate}[a.]
		\item Tabel LOG\_TRX\_PEMBAYARAN
		
\begin{lstlisting}
                                                                                                                                            
\end{lstlisting}
		
		\item Tabel LOG\_REVERSAL
		
\begin{lstlisting}
\end{lstlisting}
		
	\end{enumerate}	
		
	\item Pembuatan Relasi Tabel
	
	Relasi dari masing-masing tabel adalah sebagai berikut :
	
	
	
	\item Pembuatan Indeks
	
	
	
	\item Pembuatan \textit{Trigger}
	\item Pembuatan \textit{Store Procedure}
\end{enumerate}

\subsection{menguji database}

\subsection{pelatihan pengguna - ops}


\section{RANCANGAN}

% terlampir dari kegiatan III.D.1

\section{LOKASI}

\section{SKEMA DAN KAMUS}

\section{BESARAN}

\section{JENIS DBMS}

\section{JENIS APLIKASI YANG MENGGUNAKAN DB}

\end{document}